\documentclass{article}
\usepackage[letterpaper, top=2cm,bottom=2cm, left=3cm, right=3cm, marginparwidth = 1.75cm]{geometry}
\usepackage[utf8]{inputenc}
\usepackage{amsmath}
\usepackage{bigints}
\usepackage{amssymb}
\usepackage[mathscr]{euscript}
\usepackage{mathabx}
\usepackage{graphicx}
\usepackage{cancel}
\usepackage{tikz}
\usepackage{pgfplots}
\pgfplotsset{compat = newest}
\usepackage{siunitx}
\usepackage{tkz-euclide}
\usepackage{pstricks}
\providecommand{\abs}[1]{\lvert#1\rvert}
\usetikzlibrary{3d}
\makeatletter
\tikzoption{canvas is plane}[]{\@setOxy#1}
\def\@setOxy O(#1,#2,#3)x(#4,#5,#6)y(#7,#8,#9)%
  {\def\tikz@plane@origin{\pgfpointxyz{#1}{#2}{#3}}%
   \def\tikz@plane@x{\pgfpointxyz{#4}{#5}{#6}}%
   \def\tikz@plane@y{\pgfpointxyz{#7}{#8}{#9}}%
   \tikz@canvas@is@plane
  }
\makeatother

\title{Justificación Proyecto 3}
\author{Equipo N}

\begin{document}
  \maketitle
  Esta simulación interactiva tiene como objetivo modelar una máquina de Carnot que funciona entre dos reservorios de temperatura $T_H$ y $T_L$, 
  la eficiencia está dada por
  \begin{center}
    $\eta = 1 - \cfrac{T_L}{T_H}$
  \end{center}
  Con $T_H = 1050$ K y $T_L = 300$ K.\\
  El barco se mueve en aguas marinas y genera trabajo a partir del gradiente de temperatura entre un reservorio de agua a una temperatura $T_H$ 
  en el barco y $T_L$ con agua del mar.\\
  Partimos de la definición de eficiencia
  \begin{center}
    $1 - \cfrac{T_L}{T_H} = \cfrac{W}{Q_H}$
  \end{center}
  Despejando para $W$
  \begin{center}
    $W = Q_H\left(1 - \cfrac{T_L}{T_H}\right) = mC\Delta T\left(1 - \cfrac{T_L}{T_H}\right)$
  \end{center}
  Como el trabajo se aplica horizontalmente sobre el barco, podemos escribir (de alguna manera, una turbina logra crear una fuerza igual en magnitud
  pero en sentido contrario sobre el agua, que mueve al barco)
  \begin{center}
    $F\Delta x = mC\Delta T\left(1 - \cfrac{T_L}{T_H}\right)$
  \end{center}
  Consideramos la situación ideal de un barco que se desplaza al aplicarle una fuerza de 1 N, y con masa de 1 kg, por lo tanto, obtenemos
  \begin{center}
    $\Delta x = \cfrac{mc\Delta T\left(1 - \cfrac{T_L}{T_H}\right)}{F}$
  \end{center}
  Dentro de la dinámica del juego incluimos objetos (carbón) a una temperatura alta (rojos) o baja (azules). En el momento en que el barco hace 
  contacto con un objeto rojo, $T_H$ aumenta 50 K y por lo tanto $\Delta x$ también aumenta, ya que
  \begin{center}
    $1 - \cfrac{T_L}{T_H} < 1 - \cfrac{T_L}{T_H + 50}$
  \end{center}
  Cuando entra en contacto con un objeto azul, $T_H$ disminuye 50 K y $\Delta x$ disminuye, de manera similar. $T_L$ se mantiene constante.

  \section*{Otras consideraciones importantes.}
  El objetivo del juego es desplazarte la mayor cantidad de $\Delta x$ como te sea posible, para esto $\Delta x$ disminuye en razón de un frame, 
  como se muestra en pantalla.\\
  El programa es base javascript-html5 y requiere abrir el archivo .html en el navegador para visualizar el proyecto.
\end{document}
